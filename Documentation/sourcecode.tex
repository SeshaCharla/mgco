The main source code spreads across the following python files namely:
\begin{enumerate}
	\item pipeline\_server.py
	\item client\_branch.py
	\item protocol.py
	\item fileilockio.py
	\item config.py
	\item basiclogging.py
\end{enumerate}
The confuguration of the application is set using 2 configuration files namely:
\begin{enumerate}
	\item GC.conf
	\item pipeline.conf
\end{enumerate}

\section{Configuration Files}

\subsection{GC.conf}
This file will contain the necessary details of the Ground Checkout servers
including hostname, port, no. of parameters and the sampling time. An example
of GC.conf is shown as below:
\lstinputlisting{../GC.conf}

\subsection{pipeline.conf}
This file contains the details required for setting up the pipeline server
including hostname, port and the sampling time. An example of pipeline.conf is
shown below:
\lstinputlisting{../pipeline.conf}


\section{Python Modules}

\subsection{basiclogging.py}
The file contains the function necessary for logging the events in various
processes.\\
$log$(message, address, log-level) : Logs the message
in to a file named after the GC server to which the client is connected to.

\subsection{config.py}
This file has 2 functions namely $get\_config()$ and $pipeline\_config()$ which
parse GC.conf and pipeline.conf respectively an return the configuration
details in proper object type and format.
%\begin{itemize}
%	\item get\_config([GC.conf]) $->$ int n, (str, int)\_array addrs, int\_array nparms,
%		int\_array st\_list
%	\item pipeline\_config([pipeline.conf]) $->$ (str, int)\_array addrs, int\_array
%		sleeptime
%\end{itemize}


\subsection{filelockio.py}
This file has 2 functions namely $filelockread()$ and $filelockwrite()$ which read
and write in to files with a lock in place while reading or writing to avoid
simultaneous reading and writhing by different processes.


\subsection{protocol.py}
This file has the necessary functions for reading and writing data in GCO
protocol. cp1252 encoding is used as '$\backslash$xbb' and '$\backslash$xcc' are represented by
corresponding hex numbers unlike utf-8 where it is split in to 2 hex nos.\\
The following are the functions and classes defined in the file:
\begin{enumerate}
	\item $get\_frame$(no. of parameters) : Randomly generates a
		frame with given no. of parameters and returns the encoded bytes.
		The type of frame is also decided randomly.
	\item $creatheader$(no. of parameters, frame type) : Returns
		the encoded header bytes depending on the time of calling and the frame
		type and the no. of parameters.
	\item $setframetype$(GC frame list) : Returns the type of frame to be
		assigned to the concatenated GC data.
	\item $GCFrame$ : Class for creating frame objects for storing current
		data.
		\begin{enumerate}
			\item $self._\_init\_\_$(no. of parameters) : Creates the following
				attributes for the object:
				\begin{itemize}
					\item $self.nparms$ : Number of parameters in the frame
					\item $self.invalid\_data$ : Invalid data stream
					\item $self.old\_timestamp$ : Previous tiem stamp of the
						data
					\item $self.type$ : Type of data
					\item $self.data$ : Data extracted from the current frame
				\end{itemize}
			\item $self.update\_frame$(frame) : Updates the object with the newly
				received frame after checking the validity.
			\item $self.valid$(Frame object) : Checks the validity of the frame.
			\item $self.set\_invalid$ : Makes the data invalid.
			\item $self.crnparms$(Frame object) : Checks if the no. of parameters
				is consistent.
		\end{enumerate}
	\item $Frame$ : Class for creating frame objects which are nothing but data
		structures to store the frame.
		\begin{enumerate}
			\item $self.\_\_init\_\_$(frame) : Initiates the following attributes:
				\begin{itemize}
					\item self.type : Type of frame.
					\item self. timestamp : Time stamp on the frame.
					\item self.data : Data in the frame.
					\item self.hnparms : Number of parameters declared in the
						frame header.
				\end{itemize}
		\end{enumerate}
	\item $geti\_tiemstamp$(frame) : Returns the time stamp from the frame.
	\item $get\_type$(frame) : Returns the type of the frame.
	\item $get\_data$(frame) : Returns the in the frame.
	\item $get\_hnparms$(frame) : Returns the number of parameters declared in
		the frame header.
\end{enumerate}



\subsection{pipeline\_server}
This file contains the PipelineServer class which creates a PipelineServer
object which connects to
